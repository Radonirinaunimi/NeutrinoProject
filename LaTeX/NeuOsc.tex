%%%%%%%%%%%%%%%%%%%%%%%%%%%%%%%%%%%%%%%%%%%%%%%%%%%%%%%%%%%%%%%%%%%%%%%%%%
%%%%%%%%%%               (Tanjona Radonirina)                 %%%%%%%%%%%%
%%%%%%%%%%%%%%%%%%%%%%%%%%%%%%%%%%%%%%%%%%%%%%%%%%%%%%%%%%%%%%%%%%%%%%%%%%
\documentclass[twocolumn,secnumarabic,amssymb, nobibnotes, aps, prd,10pt]{revtex4-1}

\newcommand{\revtex}{REV\TeX\ }
\newcommand{\classoption}[1]{\texttt{#1}}
\newcommand{\macro}[1]{\texttt{\textbackslash#1}}
\newcommand{\m}[1]{\macro{#1}}
\newcommand{\env}[1]{\texttt{#1}}
\setlength{\textheight}{9.5in}

\usepackage{slashed}
\newcommand{\ignore}[1]{}

\usepackage{amsmath,multirow}
\setcounter{secnumdepth}{3}

\usepackage{color}
\definecolor{azure}{rgb}{0.0, 0.44, 1.0}
\definecolor{ceruleanblue}{rgb}{0.16, 0.32, 0.75}

\usepackage{hyperref}
\hypersetup{
	colorlinks=true,
	linkcolor=red,
	citecolor=azure,
	urlcolor=cyan,
}

\usepackage{graphicx}
\usepackage{wrapfig}
%\usepackage{sidecap}
\usepackage{subcaption}	 
\usepackage{lipsum}  

% Define path to plots
\graphicspath{{../pythonCode/analysis/plots/Osc3Neutrinos/}}

\newcommand{\kt}[1]{\vert #1 \rangle}
\newcommand{\bt}[2]{\langle #1 \vert #2 \rangle}
\newcommand{\bkt}[3]{\langle #1 \vert #2 \vert #3 \rangle}

\newcommand{\Eq}[1]{Eq.$\:$(\ref{#1})}
\newcommand{\myref}[1]{Ref.$\:$\cite{#1}}
\newcommand{\Fig}[1]{Fig.$\:$\ref{#1}}
\newcommand{\App}[1]{App.$\:$\ref{#1}}
\newcommand{\SubApp}[2]{App.$\:$\ref{#1}.\ref{#2}}
\newcommand{\Sec}[1]{Sec.$\:$\ref{#1}}
%************************************************************************************************************

\begin{document}

\title{\texorpdfstring{Phenomenological Introduction to Neutrino Oscillations}{Phenomenological Introduction to Neutrino Oscillations}}

\author{Tanjona R.\ Rabemananjara}
\email{tanjona.rabemananjara@mi.infn.it}
\affiliation{Universita degli Studi di Milano \\
Via Celoria 16, Milano 20133, Italy}

\maketitle
%\tableofcontents

%%%%%%%%%%%%%%%%%%%%%%%%%%%%%%%%%%%%%%%%%%%%%%%%%%%%%%%%%%%%%%%%%%%%%%%%%%%%%%%%%%%%%%%%%%%%%%%%%%%%%%
\section{Introduction}



\section{Experimental evidence of neutrino oscillation phenomenon}
\label{sec:experiment}


\section{Theoretical framework}
\label{sec:theory}

Neutrinos interacts with other particles via weak interactions which are described
by the charge current (CC) and neutral current (NC) interaction Lagrangians:
\begin{align}
\mathcal{L}_{CC} &= - \frac{g}{2 \sqrt{2}} \mathcal{J}^{CC}_\rho W^\rho + h.c, \\
\mathcal{L}_{NC} &= - \frac{g}{2 \cos(\theta_W)} \mathcal{J}^{NC}_\rho Z^\rho,
\end{align}
where $g$ represents the $SU(2)_L$ gauge coupling constant and $\theta_W$ the weak
angle. Furthermore, the charged and neutral current that are respectively denoted
by $\mathcal{J}^{CC}_\rho$ and $\mathcal{J}^{NC}_\rho$ are defined as:
\begin{align}
\mathcal{J}^{CC}_\rho &= 2 \sum_{l} \bar{\nu}_{lL} \gamma_\rho l_L, \\
\mathcal{J}^{NC}_\rho &= \sum_{l} \bar{\nu}_{lL} \gamma_\rho \nu_{lL},
\end{align}
where the charged leptonic field $l$ can be one of the neutrino flavors $(e, \mu, \tau)$.

If the neutrinos have zero-masses, then the left-handed neutrino field $\nu_{\alpha L}$
with flavor $\alpha$ can be written as a superposition of the left-handed components
$\nu_{iL}$ of the neutrino field with mass $m_i$. In an ultra-relativistic scenario, we
have:
\begin{align}
\nu_{\alpha L} = U_{\alpha i} \nu_{i L}
\label{eq:superposition}
\end{align} 
where repeated indices are summed over. Henceforth, we use Greek letters to refer to
the neutrino masses and Latin letters to refer to the flavours. The neutrino masses
$i$ runs from $1$ to $N$ where $N$ denotes the number of massive neutrinos. In this
project, we are only going to focus in the case $N=3$.

In \Eq{eq:superposition}, $U$ is a unitary matrix. This implies that a flavor eigenstate
$\kt{\nu_\alpha}$ can be written as a superposition of different mass eigenstates $\kt{\nu_i}$
in the following way:
\begin{align}
\kt{\nu_{\alpha L}} &= U^\star_{\alpha i} \kt{\nu_i}, \\
\kt{\bar{\nu}_{\alpha L}} &= U_{\alpha i} \kt{\bar{\nu}_i}.
\end{align}
Assuming that we have three $(03)$ massive neutrinos and that the neutrinos are Dirac
particles, the mixing unitary matrix $U$ can be written as:
\begin{align}
U = \cdots ,
\label{eq:evolution_matrix}
\end{align}
where $c_{ij} = \cos(\theta_{ij})$ and $s_{ij} = \sin(\theta_{ij})$. Here, $\theta_{ij}$
denote the mixing angles while $\delta$ denotes the Dirac-type CP-phase. Defining 
$\Delta_{ij} = m_i^2 - m_j^2$ and ordering the masses such that $\Delta_{21}^2 > 0$
and $\Delta_{21}^2 < \Delta_{31}^2$, we have the following constraints:
\begin{align}
0 \leq \theta_{ij} \leq \frac{\pi}{2} \quad (i \neq j), \quad 0 \leq \delta \leq 2 \pi .
\end{align}


\subsection{Neutrino evolution equation in vacuum and in matter}
\label{subsec:evoleq}

The evolution equation of a generic neutrino state $\kt{\nu(t)}$ is described by a
Shrödinger-like equation:
\begin{align}
\mathrm{i} \partial_t \kt{\nu(t)} = H \kt{\nu(t)},
\end{align}
where $H$ represents the Hamiltonian operator. Expressed in the flavor eigenstate
basis $\kt{\nu_\alpha}$, the above equation translates into
\begin{align}
\mathrm{i} \partial_t v^{(f)}(t) = H^{(f)} \nu^{(f)}(t),
\end{align}
where $v^{(f)}(t)$ denotes the vector describing the flavor content of the neutrino state
$\kt{\nu(t)}$. Elements of the Hamiltonian matrix $H^{(f)}$ are given by
\begin{align}
H^{(f)}_{\alpha \beta} = \bkt{\nu_\alpha}{H}{\nu_\beta}.
\end{align}
In the mass eigenstate basis, the vacuum Hamiltonian $H^{(m)}$ (where $m$ indicates the
mass eigenstate representation) is determined in terms of the neutrino masses
\begin{align}
H^{(m)}_{vac} &= \mathrm{diag} \left( \sqrt{\vec{p}^2 + m_1^2}, \sqrt{\vec{p}^2 + m_2^2},
\sqrt{\vec{p}^2 + m_3^2} \right), \nonumber \\
&\approx \vert \vec{p} \vert + \frac{1}{2 \vert \vec{p} \vert} \mathrm{diag} \left( m_1^2, 
m_2^2, m_3^2 \right).
\end{align}
In the first equality we assumed that the neutrino state $\kt{\nu (t)}$ can be described
as a superposition of states with fixed momentum $\vec{p}$. In the last line, we used the
ultra-relativistic approximation $\sqrt{\vec{p}^2 + m_i^2} \sim \vert \vec{p} \vert + m_i^2
/2 \vert \vec{p} \vert$. The new Hamiltonian in the flavor eigenstates therefore reads
\begin{align}
H^{(f)}_{\alpha \beta, vac} = U_{\alpha i} H^{(m)}_{ij, vac} U^\dagger_{i \beta}
\end{align}

In the presence of matter, we have to add the vacuum Hamiltonian an effective potential
$V$ in the evolution equation
\begin{align}
i \partial_t \kt{\nu (t)} = \left( H_{vac} + V \right) \kt{\nu (t)},
\end{align}
where in the context of SM, the effective potential is a matrix that is diagonal in the
flavor basis $V^{(f)} = \mathrm{diag} \left( V_e, V_\nu, V_\tau \right)$.


\subsection{Survival probability}
\label{subsec:survival}

Assuming that a neutrino is created a a time $t_0 = 0$ at a point $x_0 = 0$, the flavor
state $\kt{\nu (t)}$ in the flavor eigenstate basis can be written as 
\begin{align}
\nu^{(f)} (0) = \left( \bt{\nu_e}{\nu(0)}, \bt{\nu_\mu}{\nu(0)}, \bt{\nu_\tau}{\nu(0)}
\right)^\mathrm{T}.
\end{align}
After some interval of time $t$ at a given point $x$, its flavor has evolved  according to
\begin{align}
\nu^{(f)} (x) = S^{(f)} (x) \nu^{(f)}(0),
\end{align}
where the evolution operator $S$ is expressed as
\begin{align}
S^{(f)} = T \left[ \mathrm{exp} \left( - \mathrm{i} \int_{0}^{x} \mathrm{d}\tilde{x} \:
H^{(f)} (\tilde{x}) \right) \right].
\end{align}
In the above equation, $\mathrm{T}$ represents the time ordering operator. Therefore, 
the probability to detect a neutrino of flavor $\nu_\beta$ at a distance $L$
from its initial position (where its initial flavor is known) is given by
\begin{align}
P (\nu_\beta \longleftarrow \nu_\alpha) = \vert S^{(f)}_{\beta \alpha (L)} \vert^2.
\label{eq:prob_s}
\end{align}
This implies that the survival probability $P (\nu_\alpha \longleftarrow \nu_\alpha)$
is given by the following
\begin{align}
P (\nu_\alpha \longleftarrow \nu_\alpha) = 1 - P (\nu_\beta \longleftarrow \nu_\alpha).
\end{align}
The above expression is a consequence of the unitarity of the mixing matrix 
$\sum_\beta P (\nu_\beta \longleftarrow \nu_\alpha) = 1$.


\subsection{Vacuum neutrino oscillations}
\label{subsec:2neutrinoOsc}

As it was presented in the previous section, in vacuum the neutrino Hamiltonian $H$ is
constant. Hence, the evolution operator can be written as
\begin{align}
S^{(f)} = U S^{(m)} U^\dagger .
\end{align}
The evolution operator in the mass eigenstate (denoted by the $(m)$) is a diagonal matrix
that is function of $\phi_i = -m_i^2 x / 2 \vert \vec{p} \vert$
\begin{align}
S^{(m)} = \mathrm{diag} \left( \mathrm{exp}(i \phi_1), \mathrm{exp}(i \phi_2), 
\mathrm{exp}(i \phi_3) \right) .
\end{align}
The probability of observing a neutrino of flavor $\nu_\alpha$ to change into a neutrino
of flavor $\nu_\beta$ is given by
\begin{align}
P (\nu_\beta \longleftarrow \nu_\alpha) = \left( U_{\beta i} U^\star_{\alpha i} 
U^\star_{\beta j} U_{\alpha j} \right) \mathrm{exp}(i \phi_{ij}),
\label{eq:prob}
\end{align}
where in the ultra-relativistic limit $\vert \vec{p} \vert \sim E$ and hence
$\phi_{ij} = \left( \Delta_{ij} L \right) / (2 E)$.
From this result, we are now equipped with the tools needed to compute the oscillation
probabilities. For the case of two neutrino flavor mixing, we only consider one 
non-vanishing mixing angle $\theta_{ij}$ in the evolution operator $U$ described
by \Eq{eq:evolution_matrix}. The oscillation probability in \Eq{eq:prob} then leads
to the following well known result
\begin{align}
P (\nu_\beta \longleftarrow \nu_\alpha) = \sin^2 (2 \theta_{ij}) \sin^2 \left(
\frac{\Delta_{ij}^2 L}{4 E} \right) .
\label{eq:2OscProb}
\end{align}
Notice that in the above expression, $\alpha$ has to be different from $\beta$ 
$(\alpha \neq \beta)$ and depending on the non-vanishing mixing angle, we end up with
different flavor changes, i.e.:
\begin{align}
\theta_{12} & \neq 0 \Longleftrightarrow P (\nu_\mu \longleftarrow \nu_e), \\
\theta_{23} & \neq 0 \Longleftrightarrow P (\nu_\tau \longleftarrow \nu_\mu), \\
\theta_{13} & \neq 0 \Longleftrightarrow P (\nu_\tau \longleftarrow \nu_e) .
\end{align}
In the three-neutrino case, by performing the same steps and using further constraint
on the masses as introduced previously $(\Delta_{21}^2 << \vert \Delta_{31}^2 \vert)$,
we have for instance, for $\nu_e \longrightarrow \nu_\mu$, the following probability
\begin{align}
P (\nu_\mu \longleftarrow \nu_e) =& s_{23}^{2} S_{23} \sin^2 (2 \theta_{13})  \nonumber \\
& + c_{23}^{2} S_{12} \sin^2(2 \theta_{12}) - 8 J S_{12} S_{13} ,
\end{align} 
where $S_{ij} = \sin^2 \left( \Delta_{ij}^2 L / (4E) \right)$ and 
\begin{align}
J = \cdots .
\end{align}


\section{Phenomenology of neutrino oscillation probabilities}
\label{sec:pheno}

In this section, 

\begin{figure}
\captionsetup[subfigure]{aboveskip=-1.5pt,belowskip=-1.5pt} 
\begin{subfigure}{1.05\linewidth}
\includegraphics[width=\linewidth]{Osc3VacuumBaseline.pdf}
\caption{Varying distance (baseline).} 
\label{higgs:sspt} 
\end{subfigure} 
\\
\begin{subfigure}{1.05\linewidth}
\includegraphics[width=\linewidth]{Osc3VacuumEnergy.pdf}
\caption{Varying energy.} 
\label{fig:vacuum} 
\end{subfigure}
\caption{Three-neutrino oscillation probabilities in vacuum with varying (a) baseline
and (b) varying energy. The top panels show the actual probabilities while the bottom
panels show the sum of each probability.}
\end{figure}

\subsection{Numerical insights}
\label{subsec:numerical}

From the analytical point of view, computing probabilities of flavor transition involve
diagonalizing the Hamiltonian operator. This procedure, however, can be complicated especially
when one studies neutrino oscillations in matter. As is often done, careful studies of
perturbative methods that lead to some approximations are used in order to computed these
probabilities. As described in \cite{}, numerical methods can bring new insights into 
understanding how these probabilities are computed. The resulting methods provide strategies
to explore non-standard oscillations where the Hamiltonians do not have generic analytical
solutions. 

In this project, we follow closely the method described in \myref{} first introduced
by Ohlsson and Snellman in \myref{}. It consists on expanding the Hamiltonian operator 
that enters in the expression of the evolution operator in terms of $SU(2)$ and $SU(3)$ 
matrices. The steps for such a calculation can be broken down into the following steps:
\begin{itemize}
\item[-] First, the Hamiltonian is expanded in terms of the Pauli matrices in the case of
two-neutrino oscillation, and in terms of the Gell-Mann matrices in the case of 
three-neutrino flavors.
\item[-] Compute the coefficients of the expansions in terms of the components of the
Hamiltonian.
\item[-] The exponent $\exp (\mathrm{i} H t)$ is then expanded using the Cayley-Hamilton theorem
which states that any analytic function of an $n \times n$ matrix can be written as a 
polynomial of degree $(n-1)$ in that matrix.
\item[-] Compute the evolution operator in terms of the coefficients in the Hamiltonian
series and derive the corresponding probability.
\end{itemize}
In this report, we only illustrate the case for two-neutrino oscillations. For the 
three-neutrino case, refer to \myref{}. Let us denote the two-neutrino Hamiltonian
by $\mathrm{H}_2$. Its expansion in terms of the Pauli matrices $\sigma^i$ is given by
\begin{align}
\mathrm{H}_2 = h_0 + h_i \sigma^i \quad \quad (i=1,2,3).
\end{align}
The coefficients $h_k$ are fully determined by the components of the Hamiltonian matrix
and can be easily computed using the explicit expression of the Pauli matrices.
\begin{align}
h_0 = 
\end{align}
The evolution operator $\mathrm{U}_2$ for the twp-neutrino oscillations case can
therefore be written as
\begin{align}
\mathrm{U}_2 = \exp \left( - \mathrm{i} (h_0 + h_i \sigma^i) L \right).
\end{align}
The first term in the exponent does not affect the probability and therefore can be 
ignored. Hence, the evolution operator just becomes $\mathrm{U}_2 = \exp (- \mathrm{i}
h_i \sigma^i L)$. Using Euler's formula, the above expression yields
\begin{align}
\mathrm{U}_2 = \cos (\vert h \vert L) -  \frac{\mathrm{i}}
{\vert h \vert} \sin (\vert h \vert L) h_i \sigma^i ,
\end{align}
where $\vert h \vert = \sqrt{ \vert h_1 \vert^2 + \vert h_2 \vert^2 +
\vert h_3 \vert^2}$. We can now compute the survival probability 
\begin{align}
P (\nu_\alpha \longleftarrow \nu_\alpha) = \vert \nu_\alpha^\dagger \mathrm{U}_\alpha
\nu_\alpha \vert^2 .
\label{eq:prob_norm}
\end{align} 
Considering $\nu_\alpha = (1,0)^\mathrm{T}$, we can compute the terms that enter in the
expression of the probability in \Eq{eq:prob_norm}
\begin{align}
\nu_\alpha^\dagger \mathrm{U}_\alpha \nu_\alpha = \cos (\vert h \vert L) - \mathrm{i}
\frac{h_3}{\vert h \vert} \sin (\vert h \vert L).
\end{align}
Putting this expression back into \Eq{eq:prob_norm}, and with some algebras we can
derive the final expression of the survival probability of a neutrino of flavor $\alpha$,
\begin{align}
P (\nu_\alpha \longleftarrow \nu_\alpha) =  \cos^2 (\vert h \vert L) +
\frac{\vert h_3 \vert^2 }{\vert h \vert^2} \sin^2 (\vert h \vert L).
\end{align}
Therefore, the expression of the two-neutrino oscillation probabilities that a neutrino
of flavor $\alpha$ is detected with a flavor $\beta$ is given by the following simple
relation:
\begin{align}
P (\nu_\beta \longleftarrow \nu_\alpha) = 1 - P (\nu_\alpha \longleftarrow \nu_\alpha).
\end{align}

Let us now show that using this approach, we re-derive the two-neutrino oscillation
probability given by \Eq{eq:2OscProb}. For two-neutrino flavor oscillation, the vacuum
Hamiltonian operator is defined as
\begin{align}
\mathrm{H}_2^{vac} = \frac{1}{2 E} \mathrm{R}_{2, \theta} \mathrm{H}^{(m)}_2 
\mathrm{R}_{2, \theta}^\dagger ,
\end{align}
where $\mathrm{R}_{2, \theta}$ represents an Euler rotation with angle $\theta_{ij}$
and $\mathrm{H}^{(m)}_2$ is given by 
\begin{align}
\mathrm{H}^{(m)}_2 = \mathrm{diag} \left( \frac{\Delta_{ij}^2}{2}, - \frac{\Delta_{ij}^2}{2} \right). 
\end{align}
Using the coefficients $h_i$ to compute $\vert h \vert$, putting the explicit expression 
of the coefficients into \Eq{eq:prob_norm}, and doing some simplification, we arrive at 
the following expression
\begin{align}
P (\nu_\beta \longleftarrow \nu_\alpha) = \sin^2 (2 \theta_{ij}) \sin^2 \left(
\frac{\Delta_{ij}^2 L}{4 E} \right) ,
\end{align} 
which is exactly the same as in \Eq{eq:prob_norm}. The difference being that this approach
can be easily implemented numerically to compute oscillations in matter. Going into full
details of using this method to compute the three-oscillation probabilities or to study
neutrino oscillations in matter is beyond the scope of this project and will be left
aside. For complete details and computations, refer to \myref{}. The next section, however,
will present results beyond the two-neutrino oscillations both in vacuum and in matter.


\subsection{Phenomenological results}
\label{subsec:results}

This section is dedicated to the phenomenological study of the three-neutrino oscillations
both in vacuum and in presence of matter with constant density. To compute the oscillation
probabilities, the only input parameters that we vary are the energy $E$ and the distance
traveled by the neutrino $L$. The other parameters such as the masses and the mixing angles
that enter in the expression of the PMNS mixing matrix are extracted from the NuFit code 
that fix the parameters to experimental data in order to find the best-fit values. These
parameters are summarized in Table \ref{tab:params}.
\begin{table}[!ht]
\centering
\begin{tabular}[t]{|c|c|}
\hline
\textbf{Parameters} 		& \textbf{Numerical Values} \\
\hline
$\delta_{CP}$			& $217^{\circ}$  \\
\hline
$\sin^2 \theta_{12}$		& $0.310$ \\
\hline
$\sin^2 \theta_{23}$		& $0.582$ \\
\hline
$\sin^2 \theta_{13}$		& $0.022$ \\
\hline
$\Delta_{21}$			& $7.391 \cdot 10^{-5}$ \\
\hline
$\Delta_{31}$			& $2.525 \cdot 10^{-3}$ \\
\hline
\end{tabular}
\caption{Numerical values of the mass differences $\Delta_{ij}$ and mixing angles
$\theta_{ij}$ extracted from a global fit to oscillation data \cite{Esteban:2018azc, }.}
\label{tab:params}
\end{table}

The plots shown in this project were produced using python codes \cite{} which in turn rely
on an external python library called \texttt{NuOsc} \cite{}.

In \Fig{fig:vacuum}, we plot the three-neutrino oscillation probabilities as a function of
the baseline $L$ and energy $E$. In particular, we plot the probability that a neutrino
with an initial flavor $\nu_e$ oscillates between flavors $\nu_\beta$ (with $\beta = e,
\mu, \tau$). We can clearly see in the bottom panels that the sum of all probabilities
exactly gives one. For small values of $L$, we see that the probability that the 
electron-neutrino does not oscillate is high. As the neutrino traverses more distances,
this probability fluctuates, and in some regions smaller than the probability of the
neutrino to have a flavor $\nu_\mu$ or $\nu_\tau$.

\begin{figure}
\captionsetup[subfigure]{aboveskip=-1.5pt,belowskip=-1.5pt} 
\begin{subfigure}{1.05\linewidth}
\includegraphics[width=\linewidth]{Osc3MatterBaseline.pdf}
\caption{Varying distance (baseline).} 
\label{higgs:sspt} 
\end{subfigure} 
\\
\begin{subfigure}{1.05\linewidth}
\includegraphics[width=\linewidth]{Osc3MatterEnergy.pdf}
\caption{Varying energy.} 
\label{fig:matter} 
\end{subfigure}
\caption{Three-neutrino oscillation probabilities in matter with varying (a) baseline
and (b) varying energy.}
\end{figure}



%%%%%%%%%%%%%%%%%%%%%%%%%%%%%%%%%%%%%%%%%%%%%%%%%%%%%%%%%%%%%%%%%%%%%%%%%%%%%%%%%%%%%%%%%%%%%%%%%%%%
%\nocite{*}
\bibliographystyle{unsrtnat}
\bibliography{biblio}


\end{document}

